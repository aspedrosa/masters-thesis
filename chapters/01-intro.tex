\chapter{Introduction}
\label{chapter:introduction}

Oftentimes medical researchers do studies associated with diseases, such as determining the impact of a certain drug or find variables that are characteristic of certain diseases.
To perform such studies and have reliable results, a great amount of data is required.
To obtain that data, these researchers have to contact medical data owners to have access to relevant data that can help improve their analysis and/or findings.

With this procedure emerges several problems for the researcher such as he has to find
institutions willing to share data and the process of contacting the data providers can
be cumbersome.
To aid in this whole process, several data hubs have been developed with the purpose of
making the process of data discovery easier.
One important aspect of such data hubs is that they present to the researcher meta
data, which is aggregations or summaries of the original data.
Metadata has the advantage that one doesn't have to deal with the anonymization process
of medical records, since only summaries of the initial data are retrieved
~\cite{egenvar, montra}.
With this dependency on the original data, emerges an important problem of data hubs
which is, metadata can easily be outdated after a small time window.
This could not raise a big problem, if the records were updated regularly, however this
rarely happens, mainly because either the update process is difficult or because
metadata has to be manually extracted and uploaded to the data hub.
A problem that still might arise from such platforms, is those different datasets very
often have different representation for the same concept or the data is organized in a
different layout.
The research is then hampered since either different approaches have to be taken to
analyze each dataset.

The \gls{ehden}~\cite{ehden} project has affiliations with several institutions, data sources and data custodians across the \gls{eu}, which the main goal is to, within a federated network~\cite{ehden-datapartners}, harmonize their data to the \gls{omop} \gls{cdm}, which was developed by \gls{ohdsi}, a multi-stakeholder, interdisciplinary and collaborative organization that brings out the value of health data through large-scale analytics~\cite{ohdsi-site}.
With a \gls{cdm}, the problem of having different representations for the same data
across distinct data sources is solved.
Researchers can now develop a single analysis method and then apply it to all gathered
datasets and these methods can be optimized for this specific data model, which allows
large-scale analytics.
Furthermore, also improves collaborative research~\cite{ohdsi-site}.
Still, within the scope of the \gls{ehden} project, the project has a database catalog~\cite{ehden-portal},
built with the MONTRA framework~\cite{montra}, where data owners fill metadata about
their data source manually, which brings the outdated problem already mentioned before.
Additionally, whenever new metadata fields are introduced, the data owners of all data
sources have to go manually update their metadata form.

\section{Objectives}
The purpose of this dissertation is to create a system that can automate the update process of metadata stored in online platforms, that aids in the procedure of finding the correct data set for a specific study.
Such a system must be able to extract metadata from the databases, for that, an agent software will be installed along with each databases' local system.
Additionally, as new software components might be developed, it is a great opportunity to try new technologies.

With that, the following goals were established:
\begin{itemize}
    \item have a platform capable of holding and displaying metadata in an intuitive and user-friendly way;
    \item develop or find a tool that extracts metadata from a database;
    \item design a system capable of sending data to an application, to keep their data up-to-date;
    \item make use of new technologies with growing popularity.
\end{itemize}

\section{Outline}
This dissertation is organized into five more chapters, which are described below.

Chapter \ref{chapter:background} intends to provide a state-of-the-art characterization associated with the work of the dissertation.
Regarding the several goals established, several solutions and approaches were studied.

In chapter \ref{chapter:metadata-visualization} a software framework used to develop platforms to store and visualize metadata is described.
Has this framework had some design flaws, the chapter also details all the improvements performed.

Chapter \ref{chapter:extraction-update} describes the entire development process around the tool to extract metadata and the metadata management system that automates the update process of metadata.
It starts by detailing all the requirements associated with such components, then describes both the architecture and implementation of both the extraction tool and the metadata management system.

The Chapter \ref{chapter:evalution} is used to show the integration of all the components developed in the previous chapters.

The last chapter, \ref{chapter:conclusion}, presents the main achievements with the work, main challenges found and future work.
