\chapter{Requirement Analysis}
\label{chapter:requirement-analysis}
% fase inicial explicar mais aprofundado as ferramentas que foram exploradas no background e vão ser utilizadas

\section{MONTRA}
%O que é o montra e o seu estado atual

% explicar de maneira mais aprefundada a paltaforma, realçando partes que devem ser alteradas ou que não estão corretas
%
\subsection{Communities}
\subsection{Questionnaires}
%   - questions
%   - excel (como as choices são passadas por virgulas)
\subsection{Fingerprints}
%   - answers
%   - submissions
%   - views
%    - validação feita toda do lado do cliente, existindo a possiblidade de ataques xss
%    - a validação builtin do django não está a ser usada

\section{ACHILLES}
% organização interna
% - implementado em R
% - diferentes maneiras de exportação (json, csv ou diramente para db)
% - a query for each analaysis

\section{What we seek/What is missing}

\subsection{... requirments}

% extração automatica
%  - controlo no owner
% publicação dos dados extraidos
%  - os dados devem ser enviados em vez de fetched
% organização das atualizações
%  - entidade central
% abilidade de ferramentas de visualização subscreverem a atualizações
% envio, pela entidade central, destas atualizações pedidas às ferramentas
