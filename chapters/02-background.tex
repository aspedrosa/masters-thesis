\chapter{Background}
\label{chapter:background}

Oftentimes medical researchers do studies associated with diseases, such as determining the impact of a certain drug or find variables that are characteristic of certain diseases.
To perform such studies and have reliable results, a great amount of data is required.
To obtain that data, these researchers have to contact medical data owners to have access to relevant data that can help improve their analysis and/or findings.

With this procedure emerges several problems for the researcher such as he has to find
institutions willing to share data and the process of contacting the data providers can
be cumbersome.
To aid in this whole process, several data hubs have been developed with the purpose of
making the process of data discovery easier.
One important aspect of such data hubs is that they present to the researcher meta
data, which is aggregations or summaries of the original data.
Metadata has the advantage that one doesn't have to deal with the anonymization process
of medical records, since only summaries of the initial data are retrieved
~\cite{egenvar, montra}.
With this dependency on the original data, emerges an important problem of data hubs
which is, metadata can easily be outdated after a small time window.
This could not raise a big problem, if the records were updated regularly, however this
rarely happens, mainly because either the update process is difficult or because
metadata has to be manually extracted and uploaded to the data hub.
A problem that still might arise from such platforms, is those different datasets very
often have different representation for the same concept or the data is organized in a
different layout.
The research is then hampered since either different approaches have to be taken to
analyze each dataset.

The \gls{ehden} project has affiliations with several institutions, data sources and data custodians across the \gls{eu}, which the main goal is to, within a federated network~\cite{ehden-datapartners}, harmonize their data to the \gls{omop} \gls{cdm}, which was developed by \gls{ohdsi}, a multi-stakeholder, interdisciplinary and collaborative organization that brings out the value of health data through large-scale analytics~\cite{ohdsi-cite}.
With a \gls{cdm}, the problem of having different representations for the same data
across distinct data sources is solved.
Researchers can now develop a single analysis method and then apply it to all gathered
datasets and these methods can be optimized for this specific data model, which allows
large-scale analytics.
Furthermore, also improves collaborative research~\cite{ohdsi-site}.
Still, within the scope of the \gls{ehden} project, the project has a database catalog~\cite{ehden-portal},
built with the MONTRA framework~\cite{montra}, where data owners fill metadata about
their data source manually, which brings the outdated problem already mentioned before.
Additionally, whenever new metadata fields are introduced, the data owners of all data
sources have to go manually update their metadata form.

To build a valuable data hub is then important to take into account how to:
\begin{itemize}
    \item extract metadata from a data source
    \item upload and update the metadata on the data hub
    \item automatize the two processes mentioned before
    \item receive and display the metadata on the data hub in a way that facilitates
        readability.
\end{itemize}

\section{Metadata Visualization Tools} \label{sec:viz-tools}

%30876434,Evaluation of repositories for sharing individual-participant data from clinical studies.
%31862012,The Systematic Review Data Repository (SRDR): descriptive characteristics of publicly available data and opportunities for research.

It will then be explored existing visualization platforms that enhance data discovery
by presenting summaries or metadata of records (data sources, datasets).

In some cases data can't be publicly available because it contains sensitive data or
simply the data owner might not want to share some portions of the data, for that the
tools analyzed should have privacy protection mechanisms, allowing to customize the
access and manipulation of data stored.

Furthermore, considering we want to improve and assist data discovery and reuse it is
important to have good data management to simplify such processes.
However, humans fail to achieve the necessary processing levels with present-day
scientific data.
It is then important that data is provided in such a way that machines can fetch,
understand, analyze and act on data.
For that the \gls{fair} Guiding Principles were established which contain a series of
considerations for data publishing to supports both human and machine operations such
as deposition, exploration, sharing and reuse~\cite{fair}.

%32620019,From Raw Data to FAIR Data: The FAIRification Workflow for Health Research.

Finally, it is preferential for such a tool to be open source since the available
solution might need some changes to solve our specific problem, and also it makes it
possible to receive contributions from the community.

\subsection{Search Method}

Regarding this subject, there was already done a systematic review of several tools that fit within the current search pool.
Its objective was to ``identify projects and software solutions that promote patient electronic health data discovery, as enablers for data reuse and advancement of biomedical and translational research''~\cite{systematic_review}.
From the final 20 systems, they captured their interoperability, what type of data they were providing and their after effect related to scientific results and improvements to better healthcare.
To perform their search they only used PubMed \link{https://pubmed.ncbi.nlm.nih.gov/} considering it indexes a substantially amount of health-care related work and provides a public \gls{api} which allows automation of the retrieval process.
The programmatic retrieval was done using the Biopython framework\link{https://biopython.org/} to query PubMed and consisted of eight steps.
First, a set of publications was retrieved using the search query (``data'' AND ``discovery'') OR ``discovery platform'' applied to the publications' titles.
From the results set a filtering process was done based on their title, excluding irrelevant publications from the following steps.
From each of the remaining publications, the 20 most similar publications were retrieved.
The same title filtering process mentioned before was applied again to the new result set.
The 5 most similar publications of each remaining publication were fetched.
The title filtering process was executed, then an abstract filtering process and finally a full-text assessment.
On all searches done the time window was set from January 2014 to September 2018.
Moreover, publications associated with molecular biology were excluded.

Initially, the same approach was taken, now considering a time window starting in November 2018.
However, after the second step, no new platforms were found.
Because of that, a variation of the search method mentioned before was undertaken.
The first step was skipped and some of the final tools presented on the systematic review were considered as a base set for the next steps.
From the 20 presented, only 8 were used on this initial set, since some tools weren't active or didn't have any feature mentioned in the previous section (FAIR and data protection).
Nonetheless, besides being discarded in terms of their metadata visualization features, 1 of these tools was analyzed for its metadata extraction features and 3 of them were analyzed for their metadata sharing network architecture, which will be detailed in the next sections.
Coming back to the search method, the 20 similar publications for each of the 8 papers considered were fetched, and the title filtering process was done, followed by an abstract filtering process and finally a full-text evaluation.
With this search method, 2 additional tools were considered, making a total of 10 tools to analyze on the section of metadata visualization tools.

\subsection*{eGenVar}

Good data management and an organized data sharing can improve work effectiveness, and increase data analysis.
One method to accomplish this is by having the data at a central repository and then provide clear and strong interfaces to interact with the data.
However, because of legal and privacy rules, not all data can be stored on a public central repository.
The software suite called the eGenVar~\cite{egenvar} data-management system, allows users to report, track, and share metadata on content, origin and history of files, without compromising privacy or security.
The tool can be seen as a metadata portfolio since it could be used to search data while the original files remain in a protected location.
Users need to have an account to access the system and once created can immediately start using the system for search operations, however, addition, deletion and update operations over content require a personal profile.
It is designed to connect current Laboratory Information Management Systems and workflow processing systems and to keep the source of data that is being processed through distinct systems at different locations.
Central to the system is a tagging process that allows users to tag data with new or pre-existing information, such as ontology terms or controlled vocabularies, at their convenience.
The system includes a server, a command-line client, other clients that can be developed in several programming languages and a web portal interface.

\subsection*{MONTRA}
Data catalogs are a good way to gather and present information of different areas, however, having to build a different web-based application for each distinct situation is not feasible.
To aid this creation, MONTRA~\cite{montra} was proposed as a flexible base architecture for composing data integration platforms, mainly associated with the biomedical field, allowing to centralize and share data originating from several and heterogeneous sources.
MONTRA can achieve this last point, by requiring the definition of a metadata skeleton within a community of data sources, which describes the original data, so different data sources following the same skeleton template will have a common representation, leading to a homogeneous representation of their metadata.
Yet, it is important to salient that the system only contains a skeleton of the underlying data sources, the original data is still stored in the cataloged sources' system.
The skeleton definition is an easy process any data custodian can do by saving it as a spreadsheet file and then submitting it through the application's web interface.
Then, the framework allows users to view, search, modify and delete information through simple forms, where access is controlled via a Role-Based Access Control system to ensure that proper access restrictions are imposed.
Also, a RESTful API is available which provides a set of programmatic endpoints which allows the creation of third-party applications on top of the framework.

This framework is in used to deploy the \gls{emif} Catalogue~\cite{emif}, a platform with the goal to be a marketplace where data owners can publish and share information about their clinical databases, allowing biomedical researchers to search for databases that meet their research needs.
Currently, the \gls{emif} Catalogue holds many distinct projects, combining, for example, data available in pan-European Electronic Health Records and Alzheimer cohorts.
Also, as mentioned at the beginning of this chapter, this framework was as well used to develop the EHDEN portal~\cite{ehden-portal}, which beyond allowing the search of databases over the EHNDEN network, makes available all tools and services built under the EHDEN project.

\subsection*{REDCap}
Realizing the need for researchers to be able to secure and easily collect and share data, a team at Vanderbilt University developed \gls{redcap}~\cite{redcap}, which allows data collecting and metadata gathering.
\gls{redcap}'s data capture tools can either be structured to work as a sequence of forms that investigators fill out as they advance through their projects or as a survey meant to be filled by research subjects.
\gls{redcap} allows visualizing collected data, providing views with basic statistical measures and chart visualizations, enabling the export of the data to several common formats.
Several features to help assure data quality are available, where Data Quality reports identify missing or incorrect values and outliers, validation errors and also allows the creation of custom rules to evaluate data correctness.
Collected Data collected be imported using the Data Import tool, furthermore, the system offers an API to support remote insertion and fetch of data.
There are also many features that enable support for various types of clinical and basic science research.
\gls{redcap} also has a collaboration functionality, enabling investigators, after adding team members to a project, to assign permissions to each based on their roles and data needs.

The tool is used by the Vanderbilt research data warehouse framework~\cite{vanderbilt}, which consists of repositories with identified and de-identified clinical data and uses tools top of its data layer to help researchers across the enterprise, \gls{redcap} being one of them.
Finally, the Ontario Brain Institute’s ''Brain-CODE``~\cite{braincode} is a platform designed to promote the collection, storage, sharing and analysis of data over several brain diseases, with the intention to understand common underlying causes of each specific dysfunction and find new ways to develop a treatment.
\gls{redcap} is one of the clinical data management systems used to
collect demographic and clinical data.

\subsection*{Data Sphere}
In the late phases of a clinical trial, scientists could retrieve a great amount of usable data about the effectiveness of certain therapeutic approaches for oncologic diseases.
With the decrease of cancer deaths over the years, another research paradigm is needed to find new or improve these therapeutic approaches.
This lead to promote data-sharing attempts to make clinical trial data accessible to the scientific research community.
The \gls{pds}~\cite{datasphere} provides a platform that meets these data-sharing needs, giving the possibility to share raw data from late-phase oncology clinical trials.
To share their data, data owners have to sign a data-sharing agreement that contains some extra data about the data they want to upload.
If this data application gets accepts, the responsibility of patient privacy is on the data providers.
Authorized users are then given the possibility to access and download all datasets made available on the platform.
To become an authorized user, the platform requires that users send an application with their background and an agreement to the terms of use.
Having these processes of sharing and getting data, prevents researchers from making different applications for each data set and allows having a more diverse data pool which can improve results from their analysis.

As of January 2021, the \gls{pds} website had available cancer trial data from over 150 trials including over 100,000 subjects \cite{datasphere-site}.

\subsection*{Molgenis}
For biologists to efficiently capture, exchange and exploit big amounts of molecular data, user-friendly and scalable software infrastructures are needed.
For that MOLGENIS~\cite{molgenis} was developed as a generic, model-driven toolkit to speed the development of custom big-data biosoftware applications.
Biological details of each biological system can be modeled using a domain-specific language, developed using XML, not requiring extensive, technical or repetitive details on how each feature should be executed, but enabling to compactly specify what kind of experiment database is desired.
MOLGENIS can also be used to create web applications to be used by biologists, tailored to their experiments, using reusable components.
The creators of MOLGENIS saw an improvement of up to 30 times in terms of efficiency when comparing to hand-writing software, besides providing several features hard to achieve by hand that were not made available by similar projects.

With the goal of developing new biomarkers and drugs, the \gls{bbmri-eric} project~\cite{bbmrieric} enables the research of basic mechanisms underlying diseases, by providing fair access to human biological samples and their associated biomedical and biomolecular data.
Here MOLGENIS is used to develop their Directory 1.0 which presents an overview of the \gls{bbmri-eric} ecosystem with its distributed structure and helps users find biobanks of their interest.

To create a centralized research resource for \gls{rd}, the RD-Connect~\cite{rdconnect} project links genomic data with patient registries, biobanks, and bioinformatics tools.
To help \gls{rd} researchers to search for \gls{rd} biobanks and registries and also inform availability and accessibility on each database's content, the RD-Connect project has the D-Connect Registry \& Biobank Finder tool, which is also a portal to other of their tools. One of them is the RD-Connect Sample Catalogue, which was developed using MOLGENIS, having an inventory of \gls{rd} biological samples available in associated biobanks.

\subsection*{Cafe Variome}
Data discovery applications connect data owners with data seekers and therefore promote data sharing, however, this last process can bring some complications.
Cafe Variome~\cite{cafevariome} is a general-purpose data discovery platform, with the goal not to be a place to store, curate or integrate information but to provide a platform to browse through the existing data.
It was developed following design principles that take into account important and emerging standards, easy to use by system administrators since is composed of a single simple software package, and also by data seekers, with flexible options allowing to customize each installation to the area of use, with a special interest on the  ''genotype-to-phenotype`` application.
To the administrator, it is given the ability to determine which data fields can be used for discovery and/or be displayed as results, and also which records are allowed to be used for discovery.

For a data seeker, the number of records that match the search term(s) are split by data source and also split into three categories:
\begin{itemize}
    \item open access: the user is allowed to see the data present on the system.
    \item linked access: the user can't see directly the data stored in the system.
        An external data source data link or data source contact information is provided.
    \item restricted access: the user has to make a data request to access the data.
\end{itemize}

\subsection*{Mica \& Opal}
Enhancing the distribution of data of epidemiological studies and making research databases interoperable are some important factors to maximize the reuse of resources and, as a consequence, promoting health developments.
Two open-source software tools, Opal and Mica~\cite{mica}, address this by providing off-the-shelf solutions for epidemiological data compatibility, management and distribution, which were proposed by Maelstrom Research.

A centralized web-based data management system is implemented with Opal, where researchers can securely import/export a wide range of data types and formats using a simple interface, which is then converted to a common model.
Nevertheless, the tool also gives the possibility to read data directly from a data source.
Privacy is ensured by storing participants' identifiers on a separate database, offering tools to manage these to administrators.
With data already imported, Opal web tools aids in data curations and quality control processes, enabling also for descriptive statistics computations to be automatized, presenting such on graphical charts.
Taking into account all these features, using Opal over multiples studies turns out to be a strong tool to standardize their epidemiological data, which then facilitates data discoverability and metadata browsing using the other tool proposed, the Mica web portal.

Mica is used to create metadata portals for single or consortia of multi epidemiological studies, with a particular enface on supporting observational cohort studies.
It is composed of several modules that allow data owners, study or network supervisors to customize detailed information about their epidemiological research datasets, studies and networks facilitating the process to organize and distribute data.
After Mica is populated with study metadata, researchers can use the built-in search engine to rapidly encounter the information that they lack for their research projects.
Furthermore, on multi-study instances, studies with a certain profile, that gather data of the desired health outcomes, risk factors and/or confounding factors are easily identified and gathered by users.
Combining Mica with one or more Opal database(s) enables users to safely query actual study data, present on remote servers, applying searches exceeding the metadata.

Besides developing these two tools, Maelstrom Research and its partners used them to develop the Maelstrom Research Catalogue~\cite{maelstrom}, a flexible collection of epidemiological studies worldwide that offer a user-friendly web-based solution for data discovery.

\subsection*{BioSharing}
Another interesting tool to promote data dissemination and discoverability is BioSharing~\cite{biosharing}, recently known as FAIRSharing \cite{fairsharing}, a portal of connected, informative and discoverable information about standards, databases, and journal and funder data policies in the life sciences.
It acts as a ``shop window'' for the three types of data mentioned, detailing relations between them, presenting metrics, as standards are developed and achieved in the databases, allowing to create a historical view, showing when certain standards are created or deprecated and when updates to a database or policy happen, giving the opportunity to data seekers to check the progression of each resource.
Currently, all records are manually curated, however many of them have been added and updated by community users, instead of FAIRSharing curators themselves.
Users are capable of claiming records for resources they manage, allowing not only to make sure that the data on their resource is valid and updated but also to gain credit for their work.
This Community curation feature, together with the coupling and enclosure of each record into standards and databases, helps FAIRSharing become a correct and complete representation of metadata standards, databases and policies in the life sciences.

\subsection*{Dataverse}
One of the tools that weren't present on the previous systematic review, and were found on the new search process was created by the Dataverse Project~\cite{dataverse}, an open-source web application to store, share, explore, analyze and cite research data.

Before the Dataverse Project, researchers had to choose between receiving credit for their data, by handling distribution themselves, however, it is difficult to have long-term preservation guarantees.
The latter could be solved by sending the data to a third-party archive system but without receiving much credit.
The Dataverse Project solves these problems: facilitates the processes of sharing data with others, also allowing to replicate others' work easily, giving the deserved credit and web visibility to all entities involved with the data being shared.
It does this by presenting a Dataverse collection (a virtual archive that contains datasets, and each dataset contains detailed metadata and data files) on the data authors' website with its original look, feel and branding along with a citation for the data.
Yet, that page is served by a Dataverse repository (an installation of Dataverse, which hosts multiple Dataverse collections), with institutional support, and long-term preservation guarantees.

\subsection*{NADA}
Another tool found on the search process performed was \gls{nada}~\cite{nada}, a web-based cataloging application that can be used as a base structure to create portals that allow users to browse, search, apply for access, compare and download census or survey data.
It was originally developed to assist the establishment of national survey data archives and is currently in use by several regional, national and international organizations.

When a \gls{nada} installation is deployed, the default catalog created is the Central Data Catalog, where all studies uploaded to \gls{nada} are visible, searchable and accessible through it.
In most cases, this is enough to have the data stored and organized, however it is possible to split the contents of this central catalog into more specific collections.
This brings advantages for both the users and the administrators since the former can more easily filter and search collections of surveys that are related, the latter can better distribute management responsibilities to specific administrators for their specific study area.

The \gls{nada} uses the Data Documentation Initiative (DDI) standard to represent the metadata for each study, where such documents are built outside the \gls{nada} application and then imported.
Users can take advantage of such metadata by performing searches about surveys in the catalog down to the variable level.

The tool also allows to include citations at the study level, pointing to works that used the data of a certain study.
These resources are convenient for researchers to see how the data have been used previously and also to show survey funders the study impact and that the data is being used for research purposes.

The level of access to the studies datasets can be controlled at the study level, allowing to have different access restrictions for distinct studies.
The available access types are:
\begin{itemize}
    \item Data not available
    \item Direct Access Data Files: no login required
    \item Public Use Data Files: login required
    \item Licensed Data Files: application required
    \item Data available in an Enclave: application required and the data is stored on the data owners site
    \item Data available at an external repository:  data is stored on the data owners site
\end{itemize}

\section{Metadata Extraction Tools}

Associated with some projects mentioned in the previous section, several tools and
processes are relevant for the task of extracting/profiling datasets.
Next are presented such tools:

\subsection*{ACHILLES}
The \gls{ohdsi} project, already mentioned at the beginning of this chapter, offers a variety of open-source tools\cite{ohdsi-tools} that can be used on several data-analytics use cases on observational patient-level data.
A common aspect of such tools is that they can all interact with one or more databases if they had adhered to the \gls{omop} \gls{cdm}, which enables them to homogenize the analytics for diverse use cases.
With this, instead of having to build an analysis from scratch, a standard template can be used, turning the process of building analyses easier, and also enhancing transparency and reproducibility.

\gls{achilles}~\cite{achilles-github} is a software tool that provides summaries and metadata of a database conforming to the \gls{cdm}, where such can be used for characterization and quality assessment of observational health databases.
It is implemented as a package written in R~\footnote{https://www.r-project.org/}, which executes a series of SQL queries over the original \gls{cdm} database to calculate all the specific summaries, which in the context of \gls{achilles}, are called analyses.
The result of these queries is then placed on a target database schema chosen by the user.
The resulting summaries are not study-specific but can be used by researchers to explore and evaluate if the contents of databases can be used on studies that they intend to perform.
However, certain summaries might reveal some information of the original data that the data owner is not willing to share or because is sensitive patient information.
With this, variations of these tools are created where only certain summaries are extracted from the data source, removing the process of having to filter the output of the \gls{achilles}.
Such a tool was developed associated with the \gls{ehden} project~\cite{peters-tool}.

\subsection*{DataMed}
Facilitating the process to find appropriate datasets is a key aspect to improve data reuse in the biomedical domain, however, it is hard to achieve due to the biomedical data complexity and volume.
DataMed's~\cite{datamed} mission is to build a data discovery index enabling users to efficiently search and access existing datasets that are spread across several repositories.

It developed a metadata ingestion pipeline that extracts, maps, and indexes by following the \gls{dats} based on community input and an analysis of existing metadata from popular repositories.
The \gls{dats} model mentioned is used to describe the metadata elements and the structure for datasets~\cite{dats}.
DataMed's pipeline was built as a horizontally scalable, message-oriented, extract-transform-load, loosely coupled distributed system, composed of a message dispatcher and one or more data processing segments.
The dispatcher functions as an orchestrator by managing the data ingestion and processing pipeline through message queues.
Each processing component performs operations on the data such as transformation, cleanup, and/or enrichment, saves the result to a document database and then notifies the dispatcher through the message queues.
Next, a pipeline specification is used by the dispatcher to decide the next step, injecting a message on the message queue of the target consumer.

Since different data sources have distinct representations, the pipeline has to abstract retrieval modes and data formats.
This is achieved by implementing different ingestors for the possible retrieval modes and data format combinations.
Each particular ingestor transforms raw data into the \gls{dats}, and they make use of data iterator(s), which allows retrieving data only when necessary, in other words, data streaming, allowing to deal with the situations where the datasets are larger than the available system memory.

An interesting and important note is that there was already some work done on mapping datasets represented in the \gls{omop} \gls{cdm} to \gls{dats}~\cite{cdm-dats}.

% Nesstar Publisher. http://www.nesstar.com/software/publisher.html/ used to build data to later import for nada
% TODO if I end up describing this tool, mention it on the NADA subsection
% http://www.nesstar.com/software/publisher.html/

%add this ? 33451426,MARMoSET - Extracting Publication-ready Mass Spectrometry Metadata from RAW Files.

\section{Metadata Network Architectures}  % TODO discuss

Considering that we have an agent that extracts or gathers metadata from a data source,
it is also important to think on how they will transfer or communicate data with the
interface that clinical researchers use.
Next it is presented some projects that design some sort of network architecture to
retrieve data from the data owners site:

% https://en.wikipedia.org/wiki/ISO/IEC_11179#cite_note-1

\subsection*{CafeVariome}
Cafe Variome~\cite{cafevariome} was designed to be used with safe or sensitive datasets.
In the latter case, there is often a need to limit which persons might access the system to undertake data discovery.
This implies laboratories connecting together in closed or semiclosed discovery networks.
These could be all-to-all networks (where each group places a discovery interface over their own data, for use by other network members) or a hub and spokes arrangement (where the network establishes just one discovery portal that federates searches across all the partner sites), or a combination of these architectures.
Regardless of the preferred arrangement, each group might wish to fine-tune which records or fields are made available for discovery by (and potentially then shared with) each other group or individual in the network.

\subsection*{GAAIN}
The primary objective of the \gls{gaain}~\cite{gaain} is to establish a virtual community for sharing Alzheimer's-related data stored in independently-operated repositories around the world.
Neuroimaging, demographic, genetic, and biologic data are integrated together while respecting the boundaries of existing repositories and protecting the ownership of shared data.
The system architecture of \gls{gaain} contains a central server that communicates with multiple client applications that are installed at the data partner sites (\gls{dpc}).
The \gls{gaain} \gls{dpc}s at each data partner site does not interfere with or consume resources of the local production system.
Since it does not have direct access to the production database, since data is locally exported into CSV files and loaded into the \gls{dpc}', it cannot disrupt the normal operations of the production system.
The data stored in the \gls{dpc} may be updated at the convenience of the data partner, and the only institutional requirement is to change local firewall configurations to allow HTTPS traffic from the central server into the data partner's network.
\gls{gaain} data partners retain complete control over their data.
Every \gls{dpc} has an “on/off switch” which provides the freedom to immediately disconnect data from the network at any time for any reason.

When \gls{gaain} investigators query the network through its web interfaces, search requests are sent to the central server.
The database in each \gls{dpc} is queried and the results are sent back to the central server where they are aggregated into the response passed on to the web interfaces.

\subsection*{PopMedNet}
Popmednet~\cite{popmednet} is a software platform designed to facilitate the creation and operation of distributed health data networks and to meet the needs of disparate data partners, coordinating center models, and researchers.
The platform is a flexible architecture and governance models enable network designs that meet the critical needs of data partners within distributed networks, including data privacy and security requirements, system security requirements, governance and operational requirements, regulatory and workflow requirements, and monitoring of network functions.
PopMedNet includes a number of features designed to facilitate research and network learning more broadly than simply distributing queries.
These features aid researchers in identifying potential collaborators, discovering prior research conducted within their network, and understanding more about the data available within their network.

The PopMedNet platform consists of two interrelated components: a web-based portal for distributing requests and administering the network, and the DataMart Client.
Both components are combined together through a publish-and-subscribe approach, that does not require any open ports, eliminating a critical security concern for data partners.
The DataMart Client is installed on a data partner end user’s local machine, behind the data partner firewall.
There is no direct external access to local data and all queries from the network portal are pulled into the local environment rather than being pushed through an open port.
The DataMart Client acts as an inbox for data partners to receive, review, and respond to queries distributed from a network portal.
This enables data partners to review the details of all requests, including request metadata such as the name and email address of the requester, a description of the request, the purpose of the request, and the request parameters.
After review, the data partner may choose to execute the query, hold it for further review, or reject it.
This asynchronous approach to querying is a feature of the system that provides the data partners with complete control over their data and all its uses.
Data partners can choose to automate many of the query processing steps, or choose to use the manual process to ensure compliance with local requirements.

\subsection*{EHR4CR}

The EHR4CR~\cite{ehr4cr} project aims to build a robust and scalable platform that will unlock data from hospital EHR systems, in full compliance with the ethical, regulatory and data protection policies and requirements of each participating country.

%Information representation
The developed solutions allow identifying patient cohorts and extracting patient-centric data using distributed EHRs/Clinical data warehouses (CDWs).

%Platform services
The access to the clinical data locally at the data endpoints consists of three logical layers: the legacy system layer (specific to the type of CDW or EHR used by the local site), the Legacy Interface layer and the EHR4CR Data source Endpoint layer.
The Legacy Interface layer deals with the complexity involved when accessing the various types of CDW and EHR systems used and the adopted strategy for translating queries against the EHR4CR information model and pivot terminology into queries that can be executed locally against the CDW or EHR system.
The EHR4CR Data source Endpoint layer is a generic layer that exposes uniform EHR4CR endpoint interfaces to other EHR4CR services and components.
Once approved, the service provider metadata is added to the central registry and exposed services and applications can be published so that platform users can discover them.

% merge the two following paragraphs into one
The current architecture description focuses on the Protocol Feasibility Scenario (PFS) and Patient identification and Recruitment Scenario (PRS).
The most interesting scenario is PFS where The main components involved are:
• Protocol feasibility tools in the form of a workbench for studying non-identifiable distributed patient data.
• An orchestration module allowing distributed execution of eligibility criteria queries.
• Endpoint (data access) services allowing eligibility criteria query execution on local clinical data warehouse facilities.

The process of querying individual data endpoints for protocol feasibility is initiated by the workbench instance on behalf of an end-user.
The workbench then submits a series of Eligibility Criteria (EC) queries to an orchestrator service instance.
The orchestrator instance identifies and invokes the data endpoints to which the EC queries are targeted.
Finally after receiving the individual EC query results, the orchestrator service instance provides a consolidated result to the workbench instance which will eventually be displayed to the end-user.

security
In order to encompass existing local data provider security policies and firewall rules, the platform supports the invocation of web services using dynamically configurable transport bindings.
Examples include asynchronous web service invocation by employing message-oriented middleware to provide the ability (from an endpoint provider perspective) to retrieve (pull) incoming queries rather than receiving these directly (push), thus avoiding the need for endpoint providers to accept incoming connections from the Internet into their local network.
This feature ensures compliance with local data provider policies, thus facilitating platform adoption, while at the same time allowing for standards-based authentication and authorization of end-users and web service clients operating on their behalf.

\subsection*{NextGen Connect}
% User guide - https://www.nextgen.com/-/media/files/nextgen-connect/nextgen-connect-310-user-guide.pdf
% NextGen Health Care
NextGen Connected Health helps many of the nation’s largest, most respected healthcare entities streamline their
care-management processes to satisfy the demands of a regulatory, competitive healthcare industry.
With NextGen Connect, NextGen Healthcare's goal is to provide the healthcare community with a secure, efficient, cost effective means of sharing health information.

% NextGen Connect
Like an interpreter who translates foreign languages into the one you understand, NextGen Connect Integration Engine translates message standards into the one your system understands.
Whenever a "foreign" system sends you a message, NextGen Connect Integration Engine’s integration capabilities expedite the following:
Filtering – reads message parameters and passes the message to or stops it on its way to the transformation stage.
Transformation – converts the incoming message standard to another standard (e.g., HL7 to XML) (Popular Medical standards supported, ex: DICOM, HL7). Such tranformations cam also be more generic and custom by adding additional javascript or Java code.
Extraction – can "pull" data from and "push" data to a database.
Routing – makes sure messages arrive at their assigned destinations.

The interfaces you configure that perform these jobs are called channels:
A channel consists of multiple connectors.
A connector is a piece of a channel that does the job of getting data into NextGen Connect (a source connector), or sending data out to an external system (a destination connector).
Every channel has exactly one source connector, and at least one destination connector.
For example you may receive data over HTTP, then write the data out to a file somewhere, and also insert pieces of the data into your custom database.

%https://github.com/nextgenhealthcare/connect


\section{Findings}

\begin{table}[H]
    \center
    \begin{tabular}{|*{6}{c |}}
\hline 
        \multirow{2}{*}{Tool Name} & \multirow{2}{*}{Open Source} & \multicolumn{2}{c|}{Visualization/Interaction} & \multirow{2}{*}{Extraction} & \multirow{2}{*}{Network} \\
\cline{3-4}
        & & Data protection  & FAIR & &   \\
\hline
        eGenVar \cite{egenvar} & {\color{green} \cmark} \repo{https://github.com/Sabryr/EGDMS} & {\color{green} \cmark} (Users + Permissions)& {\color{green} \cmark} & {\color{red} \xmark} & {\color{red} \xmark} \\
\hline
        MONTRA \cite{montra} & {\color{green} \cmark} \repo{https://github.com/bioinformatics-ua/montra} & {\color{green} \cmark} (Role based) & {\color{green} \cmark} & {\color{red} \xmark} &  {\color{red} \xmark} \\
\hline
        REDCap \cite{redcap} & {\color{red} \xmark} & {\color{green} \cmark} (Role based) & {\color{green} \cmark} & {\color{red} \xmark} & {\color{red} \xmark}  \\
\hline
        Data Sphere \cite{datasphere} & {\color{red} \xmark} & {\color{green} \cmark} (Authorized Users Only) & {\color{red} \xmark} & {\color{red} \xmark} & {\color{red} \xmark} \\
\hline
        MOLGENIS \cite{molgenis} & {\color{green} \cmark} \repo{https://github.com/molgenis/molgenis} & {\color{green} \cmark} (Role based) & {\color{red} \xmark} & {\color{red} \xmark} & {\color{red} \xmark} \\
\hline
        Cafe Variome \cite{cafevariome} & {\color{red} \xmark} & {\color{green} \cmark} (Role based) & {\color{green} \cmark} & {\color{red} \xmark} & {\color{green} \cmark} \\
\hline
        Mica \& Opal \cite{mica} & {\color{green} \cmark} \repo{https://github.com/obiba/mica2} & {\color{red} \xmark} & {\color{green} \cmark} & {\color{red} \xmark} & {\color{red} \xmark} \\
\hline
        BioSharing \cite{biosharing} & {\color{green} \cmark} \repo{https://github.com/FAIRsharing/fairsharing.github.io/} & {\color{red} \xmark} & {\color{green} \cmark} & {\color{red} \xmark} & {\color{red} \xmark} \\
\hline
        Dataverse \cite{dataverse} & {\color{green} \cmark} \repo{https://github.com/IQSS/dataverse} & {\color{green} \cmark} (Role Based) & {\color{green} \cmark} & {\color{red} \xmark} & {\color{red} \xmark} \\
\hline
        NADA \cite{nada} & {\color{green} \cmark} \repo{https://github.com/ihsn/nada} & {\color{green} \cmark} (Access Request) & {\color{red} \xmark} & {\color{red} \xmark} & {\color{red} \xmark} \\
\hline
\hline
        ACHILLES \cite{achilles-github} & {\color{green} \cmark} \repo{https://github.com/OHDSI/Achilles/} & \multicolumn{2}{c|}{\color{red} \xmark} & {\color{green} \cmark} & {\color{red} \xmark} \\
\hline
        DataMed \cite{datamed} & {\color{green} \cmark} \repo{https://github.com/biocaddie} & \multicolumn{2}{c|}{\color{red} \xmark} & {\color{green} \cmark} & {\color{red} \xmark} \\
\hline
\hline
        GAAIN \cite{gaain} & {\color{red} \xmark} & \multicolumn{2}{c|}{\color{red} \xmark} & {\color{red} \xmark} & {\color{green} \cmark} \\
\hline
        PopMedNet \cite{popmednet} & {\color{red} \xmark} & \multicolumn{2}{c|}{\color{red} \xmark} & {\color{red} \xmark} & {\color{green} \cmark} \\
\hline
        EHR4CR \cite{ehr4cr} & {\color{red} \xmark} & \multicolumn{2}{c|}{\color{red} \xmark} & {\color{red} \xmark} & {\color{green} \cmark} \\
\hline
        NextGen Connect \cite{} & {\color{green} \cmark} \repo{https://github.com/nextgenhealthcare/connect} & \multicolumn{2}{c|}{\color{red} \xmark} & {\color{red} \xmark} & {\color{green} \cmark} \\
\hline
\end{tabular}
\end{table}

% Similarity between extraction tools -> all converge different data model to a comon one

As we can see there is no tool that all the cheks
