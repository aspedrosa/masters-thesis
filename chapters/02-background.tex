\chapter{Background}
\label{chapter:background}

From a medical standpoint, to perform their studies, researches need to contact data
owners to have access to relevant data that can help improve their analysis and/or
findings that can be applied to real cases.
With this procedure emerges several problems for the researcher such as he has to find
institutions willing to share data and the process of contacting the data providers can
be cumbersome.
To aid in this whole process, several data hubs have been developed with the purpose of
making the process of data discovery easier.
One important aspect of such data hubs is that they present to the researcher meta
data, which is aggregations or summaries of the original data.
Metadata has the advantage that one doesn't have to deal with the anonymization process
of medical records, since only summaries of the initial data are retrieved
\cite{egenvar, montra}.
With this dependency on the original data, emerges an important problem of data hubs
which is, metadata can easily be outdated after a small time window.
This could not raise a big problem, if the records were updated regularly, however this
rarely happens, mainly because either the update process is difficult or because
metadata has to be manually extracted and uploaded to the data hub.
A problem that still might arise from such platforms, is those different datasets very
often have different representation for the same concept or the data is organized in a
different layout.
The research is then hampered since either different approaches have to be taken to
analyze each dataset.

The \gls{ehden} project has affiliations with several institutions, data sources and
data custodians across the \gls{eu}, which the main goal is to, within a federated
network, harmonize their data to the \gls{omop} \gls{cdm}\cite{ehden-datapartners}.
With a \gls{cdm}, the problem of having different representations for the same data
across distinct data sources is solved.
Researchers can now develop a single analysis method and then apply it to all gathered
datasets and these methods can be optimized for this specific data model, which allows
large-scale analytics.
Furthermore, also improves collaborative research \cite{ohdsi-site}.
Still, within the scope of the \gls{ehden} project, the project has a database catalog,
built with the Montra framework \cite{montra}, where data owners fill metadata about
their data source manually, which brings the outdated problem already mentioned before.
Additionally, whenever new metadata fields are introduced, the data owners of all data
sources have to go manually update their metadata form.

To build a valuable data hub is then important to take into account how to:
\begin{itemize}
    \item extract metadata from a data source
    \item upload and update the metadata on the data hub
    \item automatize the two processes mentioned before
    \item receive and display the metadata on the data hub in a way that facilitates
        readability.
\end{itemize}

\section{Metadata visualization tools}
It will then be explored existing visualization platforms that enhance data discovery
by presenting summaries or metadata of records (data sources, datasets).

In some cases data can't be publicly available because it contains sensitive data or
simply the data owner might not want to share some portions of the data, for that the
tools analyzed should have privacy protection mechanisms, allowing to customize the
access and manipulation of data stored.

Furthermore, considering we want to improve and assist data discovery and reuse it is
important to have good data management to simplify such processes.
However, humans fail to achieve the necessary processing levels with present-day
scientific data.
It is then important that data is provided in such a way that machines can fetch,
understand, analyze and act on data.
For that the \gls{fair} Guiding Principles were established which contain a series of
considerations for data publishing to supports both human and machine operations such
as deposition, exploration, sharing and reuse \cite{fair}.

Finally, it is preferential for such a tool to be open source since the available
solution might need some changes to solve our specific problem, and also it makes it
possible to receive contributions from the community.

\subsection{Search Method}

Regarding this subject, there was already done a systematic review of several tools
that fit within the current search pool.
Its objective was to "identify projects and software solutions that promote patient
electronic health data discovery, as enablers for data reuse and advancement of
biomedical and translational research" \cite{systematic_review}.
From the final 20 systems, they captured their interoperability, what type of data they
were providing and their after effect related to scientific results and improvements to
better healthcare.
To perform their search they only used PubMed
\footnote{https://pubmed.ncbi.nlm.nih.gov/} considering it indexes a substantially
amount of health-care related work and provides a public \gls{api} which allows
automation of the retrieval process.
The programmatic retrieval was done using the Biopython
framework\footnote{https://biopython.org/} where all search queries were limited to the
time windows between January 2014 to September 2018.

A softer version of the previous systematic review was done now within November 2018
and December 2020.
Also, instead of doing a query to PubMed and then find related publications, the
process taken here was to skip the first step and find related publications of the
final 20 selected on the systematic review and then try to find some software solutions
of interest.

\subsection{Findings}

From the systematic review mentioned before, only software solutions were considered
and projects were excluded.

\begin{tabular}{ | c | c | c | c | c | }
\hline 
Tool Name & Open Source & \makecell{Warehouse \\ vs \\ Owner's site}  & Data & FAIR\\
\hline
REDCap \cite{redcap} & No & Warehouse & -- & -- \\
\hline
Vanderbilt \cite{vanderbilt} & -- & Warehouse & -- & -- \\
\hline
Data Sphere \cite{datasphere} & -- & Warehouse & -- & -- \\
\hline
BBMRI-ERIC \cite{bbmrieric} & -- & Warehouse & -- & -- \\
\hline
Brain-CODE \cite{braincode} & Yes & Warehouse & -- & -- \\
\hline
B-CAN \cite{bcan} & -- & Warehouse & -- & -- \\
\hline
RD-Connect \cite{rdconnect} & -- & Warehouse & -- & Yes \\
%\hline
%CoMetaR \cite{cometar} & -- & Warehouse & -- & -- \\
\hline
\makecell{Global Alzheimer's \\Association Interactive\\ Network} \cite{gaain} & -- & Warehouse & -- & -- \\
\hline
Cafe Variome \cite{cafevariome} & No & Both & -- & -- \\
\hline
MONTRA \cite{montra} & Yes & Warehouse & -- & -- \\
\hline
Harvest \cite{harvest} & Yes & Warehouse & -- & -- \\
\hline
eGenVar \cite{egenvar} & -- & Warehouse & -- & -- \\
\hline
PopMedNet \cite{popmednet} & -- & Warehouse & -- & -- \\
\hline
Cataloguing toolkit by Maelstrom \cite{maelstrom} & Yes & Warehouse & -- & -- \\
\hline
DataMed \cite{datamed} & Yes & -- & -- & -- \\
\hline
EHR4CR \cite{ehr4cr} & -- & -- & -- & -- \\
\hline
YummyData \cite{yummydata} & Yes & Warehouse & -- & Yes \\
\hline
BioSharing \cite{biosharing} & -- & -- & -- & -- \\
\hline
Open PHACTS \cite{phacts} & -- & -- & -- & -- \\
\hline
\end{tabular}

\section{Fingerprinting Tools}

OMOP CDM
OHDSI TOOLS 

ACHILLES

https://github.com/EHDEN/CatalogueExport

\section{Network of Fingerprinting agents}
