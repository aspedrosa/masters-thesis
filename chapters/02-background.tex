\chapter{Background}
\label{chapter:background}

From a medical standpoint, to perform their studies, researches need to contact data
owners to have access to relevant data that can help improve their analysis and/or
findings that can be applied to real cases.
With this procedure emerges several problems for the researcher such as he has to find
institutions willing to share data and the process of contacting the data providers can
be cumbersome.
To aid in this whole process, several data hubs have been developed with the purpose of
making the process of data discovery easier.
One important aspect of such data hubs is that they present to the researcher meta
data, which is aggregations or summaries of the original data.
Metadata has the advantage that one doesn't have to deal with the anonymization process
of medical records, since only summaries of the initial data are retrieved
\cite{egenvar, montra}.
With this dependency on the original data, emerges an important problem of data hubs
which is, meta data can easily be outdated after a small time window.
This could not raise a big problem, if the records were updated regularly, however this
rarely happens, mainly because either the update process is difficult or because
metadata has to be manually extracted and uploaded to the data hub.
A problem that still might arise from such platforms, is that different datasets very
often have different representation for the same concept or the data is organized in a
different layout.
The research is then hampered since either different approaches have to be taken to
analyse each dataset.

The \gls{ehden} project has affiliations with several institutions, data sources and
data custodians across the \gls{eu}, which the main goal is to, within a federated
network, harmonise their data to the \gls{omop} \gls{cdm}. % TODO cite https://www.ehden.eu/datapartners/
With a \gls{cdm}, the problem of having different representation for the same data
across distinct data sources is solved.
Researchers can now develop a single analysis method and then apply to all gathered
datasets.
Furthermore, 

To build a valuable data hub is then important to take into account how to:
\begin{itemize}
    \item extract metadata from a data source
    \item upload and update the metadata on the data hub
    \item automatize the two processes mentioned before
    \item receive and display the metadata on the data hub
\end{itemize}


\section{Metadata visualization tools}
guidelines for Data Sharing: https://doi.org/10.1177/0002764218784991

tools shold respect fair principles \cite{fair}.

\subsection{Search Method}

There was already done a systematic review of such tools \cite{systematic_review}, from
2014 until September 2018.  The same method was applied but now within the date
November 2018 until October/November 2020.  On the first related step 4 were excluded.

\subsection{Findings}

\begin{tabular}{ | c | c | c | c | c | }
\hline 
Tool Name & Open Source & \makecell{Warehouse \\ vs \\ Owner's site}  & Data & FAIR\\
\hline
REDCap \cite{redcap} & No & Warehouse & -- & -- \\
\hline
Vanderbilt \cite{vanderbilt} & -- & Warehouse & -- & -- \\
\hline
Data Sphere \cite{datasphere} & -- & Warehouse & -- & -- \\
\hline
BBMRI-ERIC \cite{bbmrieric} & -- & Warehouse & -- & -- \\
\hline
Brain-CODE \cite{braincode} & Yes & Warehouse & -- & -- \\
\hline
B-CAN \cite{bcan} & -- & Warehouse & -- & -- \\
\hline
RD-Connect \cite{rdconnect} & -- & Warehouse & -- & Yes \\
%\hline
%CoMetaR \cite{cometar} & -- & Warehouse & -- & -- \\
\hline
\makecell{Global Alzheimer's \\Association Interactive\\ Network} \cite{gaain} & -- & Warehouse & -- & -- \\
\hline
Cafe Variome \cite{cafevariome} & No & Both & -- & -- \\
\hline
MONTRA \cite{montra} & Yes & Warehouse & -- & -- \\
\hline
Harvest \cite{harvest} & Yes & Warehouse & -- & -- \\
\hline
eGenVar \cite{egenvar} & -- & Warehouse & -- & -- \\
\hline
PopMedNet \cite{popmednet} & -- & Warehouse & -- & -- \\
\hline
Cataloguing toolkit by Maelstrom \cite{maelstrom} & Yes & Warehouse & -- & -- \\
\hline
DataMed \cite{datamed} & Yes & -- & -- & -- \\
\hline
EHR4CR \cite{ehr4cr} & -- & -- & -- & -- \\
\hline
YummyData \cite{yummydata} & Yes & Warehouse & -- & Yes \\
\hline
BioSharing \cite{biosharing} & -- & -- & -- & -- \\
\hline
Open PHACTS \cite{phacts} & -- & -- & -- & -- \\
\hline
\end{tabular}

\section{Fingerprinting Tools}

ACHILLES

https://github.com/EHDEN/CatalogueExport

\section{Network of Fingerprinting agents}
