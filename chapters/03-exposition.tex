\chapter{Metadata Visualization}
\label{chapter:metadata-visualization}

\section{MONTRA}
\begin{itemize}
    \item O que é o montra e o seu estado atual
    \item explicar de maneira mais aprefundada a paltaforma, realçando partes que devem ser alteradas ou que não estão corretas
\end{itemize}

\subsection{Communities}
\subsection{Questionnaires}
\begin{itemize}
    \item questions
    \item tipo de questoes
\end{itemize}

\subsection{Fingerprints}
\begin{itemize}
    \item answers
    \item submissions (FingerprintHead)
    \item Views
    \item validação feita toda do lado do cliente, existindo a possiblidade de ataques xss
    \item a validação builtin do django não está a ser usada
\end{itemize}

\subsection{Import Questionnaires - Excel}
\begin{itemize}
    \item Como os vários conceitos anteriores são mapeados para o excel
    \item Question Sets
    \item Questions
    \item Choices
    \item ...
\end{itemize}

\subsection{Data Models}
\begin{itemize}
    \item Diagrama de classes
    \item Principal intuito de cada class
\end{itemize}

\section{Refactoring}
\begin{itemize}
    \item o que necessita de, ou vai, ser alterado
\end{itemize}

\subsection{Data Models}
\begin{itemize}
    \item explicar escolha de apenas reformular apenas a apartir de determinado nivel
    \item explicar os novos modelos e de que maneiras resolvem problemas que existiam
    \item trade offs tidos em conta
    \item diagrama de classes com diferenças
    \item tipos de perguntas novos e deprecated
\end{itemize}

\subsection{Questionnaire}
\begin{itemize}
    \item UI changes
    \item consequencia da alteração do front end devido à alateração do backend
    \item passar toda a verificação para o backend, passando a usar a validação built in do Django
    \item tentar reutilizar os modulos existentes
\end{itemize}

\subsection{Excel}
\begin{itemize}
    \item mais concreto, e tem mais em conta o contexto em volta das questoes
    \item entanto é mais estenso
\end{itemize}
